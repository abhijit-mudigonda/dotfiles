\documentclass[12pt]{article}

%Among other things, useful for typing words or names in other languages
\usepackage[T1]{fontenc}

\usepackage[utf8]{inputenc}

%Page margins
\usepackage[margin=1in]{geometry}
%Get all the standard math/theorem/symbol things
\usepackage{amsthm,amsmath,amssymb}

%For drawing images
\usepackage{tikz,xcolor,graphicx}
%For drawing commutative diagrams
\usepackage{tikz-cd}

%Arrows with superscripts for exact sequences
\usepackage{extarrows}

%Lots of control over headers and footers
\usepackage{fancyhdr}

%Good tables
\usepackage{booktabs}

%Biblatex lets you do bibliographies!
\usepackage[citestyle=alphabetic, backend=biber, bibstyle=authortitle]{biblatex}

%Lets you do source code and code snippets
\usepackage{listings}
\lstset{
    basicstyle=\small\ttfamily,
    keywordstyle=\color{blue},
    language=python,
    xleftmargin=16pt,
}

%Custom labels on enumerate, doesn't seem to work properly
\usepackage[shortlabels]{enumitem}

%Line spacing
%\usepackage{setspace}\onehalfspacing


\usepackage{comment}

%Use nicefrac for fractions in exponents and other places where fractions look ugly
\usepackage{nicefrac}

%hypertext links in document
\usepackage[unicode]{hyperref}
\hypersetup{colorlinks=true,urlcolor=blue,citecolor=blue,linkcolor=blue}

%lets you do certain linebreaks?
\usepackage{url}


%Shows internal link keys, remove in final version 
\usepackage[color]{showkeys} %add in 'final' into parameter to remove showkeys
\renewcommand\showkeyslabelformat[1]{\scalebox{.8}{\normalfont\footnotesize\ttfamily#1}\hspace{-.5em}}
% showkeys font
\colorlet{refkey}{orange!20}
\colorlet{labelkey}{blue!80}

%Use \todo to add notes
\usepackage[colorinlistoftodos]{todonotes}

%Use \marginnote to add notes on the margin
\usepackage{marginnote}
\renewcommand*{\marginfont}{\footnotesize\sffamily}

%tikz stuff
\usetikzlibrary{calc}

%% For drawing graphs (added by Abhijit Mudigonda)
\usetikzlibrary{arrows, shapes} 
\tikzstyle{vertex}=[circle,fill=black!25,minimum size=20pt,inner sep=0pt]
\tikzstyle{selected vertex} = [vertex, fill=red!24]
\tikzstyle{edge} = [draw,thick,-]
\tikzstyle{weight} = [font=\small]
\tikzstyle{selected edge} = [draw,line width=5pt,-,red!50]
\tikzstyle{ignored edge} = [draw,line width=5pt,-,black!20]

%% For figure illustration (added by Evan Chen)
%\usepackage{asymptote}
\usepackage{float}
\usepackage{pgfplots}
\usepgfplotslibrary{fillbetween}
\graphicspath{ {images/} }

% ------   Theorem Styles -------

\theoremstyle{plain}
\newtheorem{theorem}{Theorem}
\newtheorem{proposition}[theorem]{Proposition}
\newtheorem{lemma}[theorem]{Lemma}
\newtheorem{claim}[theorem]{Claim}
\newtheorem{corollary}[theorem]{Corollary}
\newtheorem{conjecture}[theorem]{Conjecture}
\newtheorem{problem}[theorem]{Problem}
\newtheorem{fact}[theorem]{Fact}

\theoremstyle{definition}
\newtheorem{definition}[theorem]{Definition}
\newtheorem{example}[theorem]{Example}
\newtheorem{exercise}[theorem]{Exercise}

\theoremstyle{remark}
\newtheorem*{remark}{Remark}
\newtheorem*{joke}{Joke}
\newtheorem*{question}{Question}

% ------   Section Styles -------
%\renewcommand{\thesection}{\thechapter.\arabic{section}} %number sections as 1.1

%\usepackage[explicit]{titlesec}
%\titleformat{\section}[block]%
%{\normalfont\bfseries\centering}%
%{\sffamily\S\thesection}{0.5em}{#1}


%\titleformat{\subsection}[block]%
%{\normalfont\bfseries}%
%{}{0.5em}{#1}%
%{\hspace{-\parindent}\color{red}\small\sffamily\S\thesubsection}{0.8em}{#1}

%\titleformat{\subsubsection}[runin]%
%{\normalfont\small\bfseries}%
%{}{0.5em}{#1}

% ------ Table of contents -----
\usepackage{titletoc}
\setcounter{tocdepth}{2}
%\setcounter{secnumdepth}{5}


% ------ Comments -----
\newif\ifnotes
\notestrue
